\newcommand{\institut}{Institut f\"ur Telekommunikationssysteme}
\newcommand{\fachgebiet}{Nachrichten\"ubertragung}
\newcommand{\veranstaltung}{Praktikum Nachrichten\"ubertragung}
\newcommand{\pdfautor}{Dirk Babendererde (321 836), Thomas Kapa (325 219)}
\newcommand{\autor}{Dirk Babendererde (321 836)\\ Thomas Kapa (325 219)}
\newcommand{\gruppe}{Gruppe:}
\newcommand{\betreuer}{Betreuer: Lieven Lange}


\newcommand{\pdftitle}{Nachrichtenuebertragung\ Praktikum\ 03}
\newcommand{\prototitle}{Praktikum 03 \\ Statistische Nachrichtentheorie}

\input{../../packages/tu_header_9}
%\begin{document}


%     \lstinputlisting{./praktikum6.sce}

%---------------------------------------------------------------------
%---------------------------------------------------------------------
%---------------------------------------------------------------------


\section{Vorbereitungsaufgaben}
\begin{quote}
    \hspace{-2em}
    \subsection{Herleitung der Spektren}
    \begin{quote}
        
        \subsubsection{Shape-Top-Abtastung}
        \begin{quote}
            
            
            \begin{equation*}
                \begin{split}
                    f_m (t)   &= f(t) \cdot \sqcap_{\alpha T} (t) \ast \delta_T (t) \\
                    F_m (j\omega) &= \frac{1}{2\pi} F (j\omega) \ast \left [ \alpha T \cdot si \left( \frac{\omega
                    \alpha T}{2} \right) \cdot \omega_T \cdot \delta_{\omega T} (\omega) \right] \\
                    &= \alpha \cdot F (j \omega) \ast \left ( si \left( \frac{\omega \alpha T}{2} \right)
                    \sum_{k=-\infty}^{\infty} \delta (\omega - k\omega_T) \right)\\
                    &= \alpha \cdot F (j \omega) \ast \sum_{k=-\infty}^{\infty} (si(k \pi \alpha) \cdot \delta (\omega -
                    k\omega_T))\\
                    &= \alpha \cdot \sum_{k=-\infty}^{\infty} \left [ si(k \pi \alpha) \cdot F (j(\omega - k\omega_T))
                    \right]\\
                \end{split}
            \end{equation*}
            
            
        \end{quote}
        
        \subsubsection{Flat-Top-Abtastung}
        \begin{quote}
            
            
            \begin{equation*}
                \begin{split}
                    f_m (t) &= [f (t) \cdot \delta_T (t)] \ast \sqcap_{\alpha T} (t)\\
                    F_m (j\omega) &= \left ( \frac{1}{2 \pi} F (j\omega) \ast \omega_T \cdot \delta_T (\omega) \right) \cdot
                    \alpha T \cdot si (\frac{\alpha \omega T}{2})\\
                    &= \alpha \cdot si \left( \frac{\omega \alpha T}{2} \right) \cdot \sum_{k=-\infty}^{\infty} F(j(\omega -
                    k\omega_T))\\
                \end{split}
            \end{equation*}
            
            
        \end{quote}
        
        
    \end{quote}
    
    
    \subsection{Skizzieren}
    \begin{quote}
        Zur übung das Spektrum folgenden Signals
        
        \begin{equation*}
        	\begin{split}
                f(t) = A \p cos(\frac{\omega_T}{5} t)
        	\end{split}
        \end{equation*}
        
        mit $\alpha = 0,5$ und $\omega_T =$ \SI{20}{\pi\ \kilo\hertz} Skizzieren.
        
    \end{quote}
    
    
    \subsection{MatLab-Simulation}
    \begin{quote}
        
    \end{quote}
    
    
    
    
    
    
\end{quote}

%--------------------------------------------------------------------
%--------------------------------------------------------------------


\section{Durchführung}
\begin{quote}
    
    
    \subsection{Labordurchführung}
    \begin{quote}
     Für das Signalausblendung (Shape-Top-Sampling) und das
     Signalverbreiterung (Flat-Top-Sampling) wird das Dual Analog Switch- Modul
     verwendet.
     Bei der beiden Samplingvarianten wird auf den Eingang Control 1 ein
     unipolaren Rechtecksignal mit einer Spannung von 0 bis 5 Volt, variabler
     Frequenz (10 und 20 Hz) und variablem Tastverhältnis (0,2 , 0,5 und 0,7)
     gelegt. Dieses wird vom Signalgenerator gelierfert.
     Es sollen drei abzutastende Signale untersucht werden. Zum einen ein
     bipolares, mittelwertfreies Rechtecksignal mit einer Spitze-Spitze-Spannung von 4 Volt und einer
     Frequenz von 2 kHz, welches vom Master Signals Modul geliefert wird. Dafür
     wird der Ausgang 2 kHz DIGITAL verwendet. Dieser liefert allerdings ein
     Rechtecksignal mit der Spannung von 0 bis 5 Volt. Damit man das
     bipolare Signal erhält, wird das ADDER Modul verwendet. Das Rechtecksignal
     wird auf den Eingang B gegeben, wo das Signal mittel der Verstärkung von
     4/5 auf 0 bis 4 Volt eingestellt wird. Da der ADDER invertiert muss zuletzt
     nur noch ein Offset von 2 Volt mit Hilfe des Variable DC Moduls gegeben
     werden.\\
     \noindent\hspace*{4mm}% 
     Als zweites Signal soll das eben erstellte bipolare Rechtecksignal noch vor
     der Abtastung tiefpassgefiltert werden. Dazu wird das RC LPF Modul, also
     ein Tiefpass bestehend aus einem Kondensator und einer Spule, verwendet.
     Zum anderen soll ein 2 kHz Sinussignal abgetastet werden, welches ebenfalls
     dem Master Signals Modul entnommen werden kann.\\
     \noindent\hspace*{4mm}% 
     Bei der Signalausblendung wird das abzutastende Signal auf den Eingang
     IN1 gegeben und der Ausgang am Out abgegriffen. Damit wird im Prinzip nur
     ein Schalter geöffnet und geschlossen, was dazu führt, dass die Kontur
     des Signal im Abtastpuls mit übertragen wird (daher Shape-Top).\\
     Bei der Signalverbreiterung wird das Quellensignal noch zusätzlich vorher
     durch ein S/H-Glied, also ein Abtast- und Halteglied (S&H
     IN, S&H OUT) und von da aus in IN1 geführt. Dieses sorgt dafür, dass bei
     einem Abtastimpuls nur ein Wert durchgängig übertragen wird (daher
     Flat-Top).\\ 
     \noindent\hspace*{4mm}% 
     Zuletzt wird das Signal zur Rekonstruktion noch durch einen Tiefpassfilter
     mit variabler Grenzfrequenz geführt (TUNEABLE LPF IN/OUT).\\
     \noindent\hspace*{4mm}% 
     Für die Messungen wird das Abtastsignal sowohl in der Frequenz, als auch in
     dem Tastgrad, wie oben gegeben für Signalverbreiterung und
     Signalausblendung) variiert.
     Im zweiten Teil soll ein Sprachsignal mit Signalausblendung oder
     Signalverbreiterung abgetastet werden. Dazu wird ein Mikrofon des SPEECH
     Modul genutzt und beim BUFFER Modul ein Kopfhörer angeschlossen.
    \end{quote}
    
    
    \subsection{Theorie \& Auswertung}
    \begin{quote}
     
    \end{quote}
    
\end{quote}









% \begin{quote}
%     \lstinputlisting[
%         caption={Scilab-script},
%         label=lst:scilab]
%         {./Scilab/Motor.sce}
%         
% \end{quote}

%--------------------------------------------------------------------
%--------------------------------------------------------------------
% \begin{thebibliography}{999}
% 
% \bibitem{Boris}Boris Henckell: Ein Paar sachen geklaut.. ähhh inspirationen geholt
% \href{http://www.krachler.com/fileadmin/user_upload/arbeiten/Reglersynthese_Christian_Krachler.pdf}{Reglersynthese nach dem Frequenzkennlinienverfahren}, S16, S22, 08.05.2012
% 
% 
% %Name, Vorname.; evtl. Name2, Vorname2.: Titel des Dokumentes
% %oder Buches, Zeitschrift/Verlag/URL (Auflage, Erscheinungsort, -jahr), ggf. Seitenzahlen
% %\bibitem [Wiki10] {DigitaleMesskette2} \url{www.wikipedia.org}, Zugriff 22.03.2010
% \end{thebibliography}


\end{document}
