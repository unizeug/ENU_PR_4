\newcommand{\institut}{Institut f\"ur Telekommunikationssysteme}
\newcommand{\fachgebiet}{Nachrichten\"ubertragung}
\newcommand{\veranstaltung}{Praktikum Nachrichten\"ubertragung}
\newcommand{\pdfautor}{Dirk Babendererde (321 836), Thomas Kapa (325 219)}
\newcommand{\autor}{Dirk Babendererde (321 836)\\ Thomas Kapa (325 219)}
\newcommand{\gruppe}{Gruppe:}
\newcommand{\betreuer}{Betreuer: Lieven Lange}


\newcommand{\pdftitle}{Nachrichtenuebertragung\ Praktikum\ 03}
\newcommand{\prototitle}{Praktikum 03 \\ Statistische Nachrichtentheorie}

\input{../../packages/tu_header_9}
%\begin{document}


%     \lstinputlisting{./praktikum6.sce}

%---------------------------------------------------------------------
%---------------------------------------------------------------------
%---------------------------------------------------------------------


\section{Vorbereitungsaufgaben}
\begin{quote}
    \hspace{-2em}
    \subsection{Herleitung der Spektren}
    \begin{quote}
        
        \subsubsection{Shape-Top-Abtastung}
        \begin{quote}
            
            
            \begin{equation*}
                \begin{split}
                    f_m (t)   &= f(t) \cdot \sqcap_{\alpha T} (t) \ast \delta_T (t) \\
                    F_m (j\omega) &= \frac{1}{2\pi} F (j\omega) \ast \left [ \alpha T \cdot si \left( \frac{\omega
                    \alpha T}{2} \right) \cdot \omega_T \cdot \delta_{\omega T} (\omega) \right] \\
                    &= \alpha \cdot F (j \omega) \ast \left ( si \left( \frac{\omega \alpha T}{2} \right)
                    \sum_{k=-\infty}^{\infty} \delta (\omega - k\omega_T) \right)\\
                    &= \alpha \cdot F (j \omega) \ast \sum_{k=-\infty}^{\infty} (si(k \pi \alpha) \cdot \delta (\omega -
                    k\omega_T))\\
                    &= \alpha \cdot \sum_{k=-\infty}^{\infty} \left [ si(k \pi \alpha) \cdot F (j(\omega - k\omega_T))
                    \right]\\
                \end{split}
            \end{equation*}
            
            
        \end{quote}
        
        \subsubsection{Flat-Top-Abtastung}
        \begin{quote}
            
            
            \begin{equation*}
                \begin{split}
                    f_m (t) &= [f (t) \cdot \delta_T (t)] \ast \sqcap_{\alpha T} (t)\\
                    F_m (j\omega) &= \left ( \frac{1}{2 \pi} F (j\omega) \ast \omega_T \cdot \delta_T (\omega) \right) \cdot
                    \alpha T \cdot si (\frac{\alpha \omega T}{2})\\
                    &= \alpha \cdot si \left( \frac{\omega \alpha T}{2} \right) \cdot \sum_{k=-\infty}^{\infty} F(j(\omega -
                    k\omega_T))\\
                \end{split}
            \end{equation*}
            
            
        \end{quote}
        
        
    \end{quote}
    
    
    \subsection{Skizzieren}
    \begin{quote}
        Zur übung das Spektrum folgenden Signals
        
        \begin{equation*}
        	\begin{split}
                f(t) = A \p cos(\frac{\omega_T}{5} t)
        	\end{split}
        \end{equation*}
        
        mit $\alpha = 0,5$ und $\omega_T =$ \SI{20}{\pi\ \kilo\hertz} Skizzieren.
        
    \end{quote}
    
    
    \subsection{MatLab-Simulation}
    \begin{quote}
        
    \end{quote}
    
    
    
    
    
    
\end{quote}

%--------------------------------------------------------------------
%--------------------------------------------------------------------


\section{Durchführung}
\begin{quote}
    
    
    \subsection{Labordurchführung}
    \begin{quote}
     
    \end{quote}
    
    
    \subsection{Auswertung \& Theorie}
    \begin{quote}
     
    \end{quote}
    
\end{quote}










% \begin{quote}
%     \lstinputlisting[
%         caption={Scilab-script},
%         label=lst:scilab]
%         {./Scilab/Motor.sce}
%         
% \end{quote}

%--------------------------------------------------------------------
%--------------------------------------------------------------------
% \begin{thebibliography}{999}
% 
% \bibitem{Boris}Boris Henckell: Ein Paar sachen geklaut.. ähhh inspirationen geholt
% \href{http://www.krachler.com/fileadmin/user_upload/arbeiten/Reglersynthese_Christian_Krachler.pdf}{Reglersynthese nach dem Frequenzkennlinienverfahren}, S16, S22, 08.05.2012
% 
% 
% %Name, Vorname.; evtl. Name2, Vorname2.: Titel des Dokumentes
% %oder Buches, Zeitschrift/Verlag/URL (Auflage, Erscheinungsort, -jahr), ggf. Seitenzahlen
% %\bibitem [Wiki10] {DigitaleMesskette2} \url{www.wikipedia.org}, Zugriff 22.03.2010
% \end{thebibliography}


\end{document}
