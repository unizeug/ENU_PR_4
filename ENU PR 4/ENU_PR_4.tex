\newcommand{\institut}{Institut f\"ur Telekommunikationssysteme}
\newcommand{\fachgebiet}{Nachrichten\"ubertragung}
\newcommand{\veranstaltung}{Praktikum Nachrichten\"ubertragung}
\newcommand{\pdfautor}{Dirk Babendererde (321 836), Thomas Kapa (325 219)}
\newcommand{\autor}{Dirk Babendererde (321 836)\\ Thomas Kapa (325 219)}
\newcommand{\gruppe}{Gruppe:}
\newcommand{\betreuer}{Betreuer: Lieven Lange}


\newcommand{\pdftitle}{Nachrichtenuebertragung\ Praktikum\ 03}
\newcommand{\prototitle}{Praktikum 03 \\ Statistische Nachrichtentheorie}

\input{../../packages/tu_header_9}
%\begin{document}


%     \lstinputlisting{./praktikum6.sce}

%---------------------------------------------------------------------
%---------------------------------------------------------------------
%---------------------------------------------------------------------


\section{Vorbereitungsaufgaben}
\begin{quote}
    \hspace{-2em}
    \subsection{Herleitung der Spektren}
    \begin{quote}
        
        \subsubsection{Shape-Top-Abtastung}
        \begin{quote}
            
            
            \begin{equation*}
                \begin{split}
                    f_m (t)   &= f(t) \cdot \sqcap_{\alpha T} (t) \ast \delta_T (t) \\
                    F_m (j\omega) &= \frac{1}{2\pi} F (j\omega) \ast \left [ \alpha T \cdot si \left( \frac{\omega
                    \alpha T}{2} \right) \cdot \omega_T \cdot \delta_{\omega T} (\omega) \right] \\
                    &= \alpha \cdot F (j \omega) \ast \left ( si \left( \frac{\omega \alpha T}{2} \right)
                    \sum_{k=-\infty}^{\infty} \delta (\omega - k\omega_T) \right)\\
                    &= \alpha \cdot F (j \omega) \ast \sum_{k=-\infty}^{\infty} (si(k \pi \alpha) \cdot \delta (\omega -
                    k\omega_T))\\
                    &= \alpha \cdot \sum_{k=-\infty}^{\infty} \left [ si(k \pi \alpha) \cdot F (j(\omega - k\omega_T))
                    \right]\\
                    \label{eq:shape}
                \end{split}
            \end{equation*}
            
            
        \end{quote}
        
        \subsubsection{Flat-Top-Abtastung}
        \begin{quote}
            
            
            \begin{equation*}
                \begin{split}
                    f_m (t) &= [f (t) \cdot \delta_T (t)] \ast \sqcap_{\alpha T} (t)\\
                    F_m (j\omega) &= \left ( \frac{1}{2 \pi} F (j\omega) \ast \omega_T \cdot \delta_T (\omega) \right) \cdot
                    \alpha T \cdot si (\frac{\alpha \omega T}{2})\\
                    &= \alpha \cdot si \left( \frac{\omega \alpha T}{2} \right) \cdot \sum_{k=-\infty}^{\infty} F(j(\omega -
                    k\omega_T))\\
                    \label{eq:flat}
                \end{split}
            \end{equation*}
            
            
        \end{quote}
        
        
    \end{quote}
    
    
    \subsection{Skizzieren}
    \begin{quote}
        Zur übung das Spektrum folgenden Signals
        
        \begin{equation*}
        	\begin{split}
                f(t) = A \p cos(\frac{\omega_T}{5} t)
        	\end{split}
        \end{equation*}
        
        mit $\alpha = 0,5$ und $\omega_T =$ \SI{20}{\pi\ \kilo\hertz} Skizzieren.
        
        
        \begin{figure}[H]
        \centering\fbox{
            \includegraphics[scale=0.7, trim = 1cm 18cm 1cm 2cm, clip]{Bilder/skizze}}
                \caption{sampling Skizze}
                \label{fig:skizze}
        \end{figure}
        
        
    \end{quote}
    
    
    \subsection{MatLab-Simulation}
    \begin{quote}
        Um unsere Durchführungsergebnisse mit idealen Werten vergleichen zu können haben wir die Praktikumsaufgaben in
        Matlab simuliert.
    \end{quote}
    
    
    
    
    
    
\end{quote}

%--------------------------------------------------------------------
%--------------------------------------------------------------------


\section{Durchführung}
\begin{quote}
    
    
    \subsection{Labordurchführung}
    \begin{quote}
     Für die Signalausblendung (Shape-Top-Sampling) und die Signalverbreiterung (Flat-Top-Sampling) wird das Dual
     Analog Switch- Modul verwendet.
     Bei beiden Samplingvarianten wird auf den Eingang Control 1 ein unipolares Rechtecksignal mit einer Spannung von 0
     bis 5 Volt, variabler Frequenz (10 und 20 Hz) und variablem Tastverhältnis (0,2 , 0,5 und 0,7) gelegt. Dieses wird
     vom Signalgenerator geliefert.
     Es sollen drei abzutastende Signale untersucht werden. Zum einen ein bipolares, mittelwertfreies Rechtecksignal mit
     einer Spitze-Spitze-Spannung von 4 Volt und einer Frequenz von 2 kHz, welches vom Master Signals Modul geliefert
     wird. Dafür wird der Ausgang 2 kHz DIGITAL verwendet. Dieser liefert allerdings ein Rechtecksignal mit der Spannung
     von 0 bis 5 Volt. Damit man das bipolare Signal erhält, wird das ADDER Modul verwendet. Das Rechtecksignal wird auf
     den Eingang B gegeben, wo das Signal mittels der Verstärkung von 4/5 auf 0 bis 4 Volt eingestellt wird. Da der
     ADDER invertiert muss zuletzt nur noch ein Offset von 2 Volt mit Hilfe des Variable DC Moduls gegeben werden.\\
     \noindent\hspace*{4mm}
     Als zweites Signal soll das eben erstellte bipolare Rechtecksignal vor der Abtastung tiefpassgefiltert werden. Dazu
     wird das RC LPF Modul, also ein Tiefpass bestehend aus einem Kondensator und einer Spule, verwendet.
     Zum anderen soll ein 2 kHz Sinussignal abgetastet werden, welches ebenfalls dem Master Signals Modul entnommen
     werden kann.\\
     \noindent\hspace*{4mm}
     Bei der Signalausblendung wird das abzutastende Signal auf den Eingang IN1 gegeben und der Ausgang am Out
     abgegriffen. Damit wird ein Schalter geöffnet und geschlossen, was dazu führt, dass die Kontur des Signals im
     Abtastpuls mit übertragen wird (daher Shape-Top).\\
     Bei der Signalverbreiterung wird das Quellensignal noch zusätzlich vorher durch ein S\&H-Glied, also ein Abtast-
     und Halteglied (S\&H IN, S\&H OUT) und von da aus in IN1 geführt. Dieses sorgt dafür, dass bei einem Abtastimpuls
     nur ein Wert durchgängig übertragen wird (daher Flat-Top).\\
     \noindent\hspace*{4mm}
     Zuletzt wird das Signal zur Rekonstruktion noch durch einen Tiefpassfilter mit variabler Grenzfrequenz geführt
     (TUNEABLE LPF IN/OUT).\\
     \noindent\hspace*{4mm}
     Für die Messungen wird das Abtastsignal sowohl in der Frequenz, als auch in dem Tastgrad, wie oben gegeben für
     Signalverbreiterung und Signalausblendung variiert.
     Dabei kann man einige der Kombinationen auslassen. Da sich die Veränderungen von $\alpha$ bei 10 und 20 kHz ähnlich
     auswirken, wird im Versuch nur der Tastgrad für 20 kHz variiert.\\
     \noindent\hspace*{4mm}
     Im zweiten Teil soll ein Sprachsignal mit Signalausblendung oder Signalverbreiterung abgetastet werden. Dazu wird
     ein Mikrofon des SPEECH Modul genutzt und beim BUFFER Modul ein Kopfhörer angeschlossen.
    \end{quote}
    
\end{quote}


\section{Auswertung \& Theorie}
\begin{quote}
    
    Im Labor haben wir dann folgende Messungen durchgefüht.
    
    \TODO{Thommy: wollen wir ne Tabelle? Neeeeiiiiiiin das sieht man dann
    doch an den Bildern}
    
    \subsection{Flat-Top Sampling}
    \begin{quote}
        
        \subsection{Änderung der Trägerfrequenz}
        \begin{quote}
            
            Der Theorie nach sollte das Signal, das mit einer höheren Frequenz abgestastet wird, auf Grund der Mehrzahl
            an Stützstellen auch besser rekonstruiert werden.
            Das reine Rechtecksignal sollte ein starkes Überschwingen zeigen, da es unendlich viele Frequenzanteile
            enthält und das Abtasttheorem für die großen Frequenzen nicht eingehlten werden kann.
            Das tiefpassgefilterte Rechtecksignal sollte aufgrund der entnommenen hohen Frequenzen weniger Überschwingen
            nach der rekonstruktion enthalten.
            Der nahezu reine Sinus sollte ziemlich gut rekonstruiert werden können, da er theoretisch nur einen
            Frequenzanteil enthält.
            
            
            Man kann gut erkennen, dass die Filterung des Rechtecksignals vor der Abtastung eine Glättung des
            rekonstruierten Signals hervorruft.
            die gemessenen Werte sehen so aus\\
            
            FIGURE\\
            \\
            
            
            verglichen mit den Simulierten werten\\
            
            FIGURE\\
            
            is das toll\\
        \end{quote}
        
        
        \subsection{Änderung des Tastgrades}
        \begin{quote}
             
             Der Theorie nach sollte das Signal, welches mit einem größeren Tastgrad abgetastet wird auch eine höhere
             Amplitude im Spektrum und im rekonstruierten Signal haben, da wie in den Formeln \ref{eq:shape} und
             \ref{eq:flat} als Vorfaktor auftauchen. Für ein kleineres $\alpha$ sollte sich aber eine bessere
             Rekonstruktion des Ausgangssignals ergeben, da das Abtastsignal immer mehr einem, für die Abtastung
             idealen, Deltaimpuls ähnelt.
             
             
        \end{quote}
    
    \end{quote}
    
    \subsection{Shape-Top Sampling}
    \begin{quote}
        
        
        \subsubsection{Änderung der Trägerfrequenz}
        \begin{quote}
            
            Auch bei der Shape-Top Abtastung 
            
        \end{quote}
        
        
        
        
        \subsubsection{Änderung des Tastgrades}
        \begin{quote}
            
             Bei der Signalausblendung sollte das Basisband (k=0) um $\alpha$ verringern. Alle anderen Bänder sollten
             sich noch zusätzlich um den Faktor $si(k\alpha\pi)$ verringern, was zu einer Verzerrung führt.
             Im Gegensatz dazu werden bei der Signalverbreiterung alle Bänder mit dem Faktor $si(\omega \alpha T/2)$
             verzerrt. Da damit das Basisbandsignal ebenfalls verzerrt ist, kann das Signal auch mit einem idealen
             Tiefpass nicht wiedergewonnen werden.
             
             Wie man erkennen kann, ähnelt die Rekonstruktion der Signalausblendung eher dem Urspungssignal, was wie
             nach der Theorie erwartet an dem verzerrten Basisband bei der Signalverbreiterung liegt.
             
        \end{quote}


        
    \end{quote}
    
    
    
    
    
\end{quote}
    









% \begin{quote}
%     \lstinputlisting[
%         caption={Scilab-script},
%         label=lst:scilab]
%         {./Scilab/Motor.sce}
%         
% \end{quote}

%--------------------------------------------------------------------
%--------------------------------------------------------------------
% \begin{thebibliography}{999}
% 
% \bibitem{Boris}Boris Henckell: Ein Paar sachen geklaut.. ähhh inspirationen geholt
% \href{http://www.krachler.com/fileadmin/user_upload/arbeiten/Reglersynthese_Christian_Krachler.pdf}{Reglersynthese nach dem Frequenzkennlinienverfahren}, S16, S22, 08.05.2012
% 
% 
% %Name, Vorname.; evtl. Name2, Vorname2.: Titel des Dokumentes
% %oder Buches, Zeitschrift/Verlag/URL (Auflage, Erscheinungsort, -jahr), ggf. Seitenzahlen
% %\bibitem [Wiki10] {DigitaleMesskette2} \url{www.wikipedia.org}, Zugriff 22.03.2010
% \end{thebibliography}


\end{document}
